\documentclass[12pt]{ucbseminar}
\usepackage{amsmath,amssymb}
\begin{document}

\name{Probability Seminar}
\organizer{Christian Gromoll \& Tai Melcher}
\dayofweek{Monday}
\room{Kerchof 326}
\time{2:00--3:00pm}

\newtalk
\date{Sep 27}
\speaker{Masha Gordina}
\affiliation{U Connecticut}
\title{Gaussian type analysis on infinite-dimensional Heisenberg groups}
\abstract{\small
Infinite-dimensional Heisenberg groups and algebras come up in a number of applications motivated by physics, including Kac-Moody algebras. At the same time they proved a nice toy model for a number of questions in analysis over infinite-dimensional curved spaces. The Heisenberg groups in question are modeled on an abstract Wiener space. Then a group Brownian motion is defined, and its
properties are studied in connection with the geometry of this group. The main results include quasi-invariance  of the heat kernel measure, log Sobolev inequality (following a bound on the Ricci curvature), and the Taylor isomorphism to the corresponding Fock space. The latter is a version of the Ito-Wiener expansion in the non-commutative setting. This is a joint work with B.Driver. 
  }

\end{document}
