
%Herbert Koch
%Institut f"ur Angewandte Mathematik
%Universit"at Heidelberg
%Email:  koch@iwr.uni-heidelberg.de
%Tel:    0049 6221 545715
%  %%%%%%%%%%%%%%%%%%%%%%%%%%%%%%%%%%%%%%%%%%%%%%%%%%%%%%%%%%%%
\documentclass[a4paper,12pt]{amsart}
\pagestyle{plain}
\oddsidemargin 1cm
\evensidemargin .8cm
\marginparsep 0pt
\topmargin .0pt
\marginparwidth 0pt
\textwidth 6in
\textheight 9.2in
\makeatletter
\renewcommand{\theenumi}{\Roman{enumi}}
\makeatother


\begin{document}

\def\R{{\mathbb R}}
\def\Z{{\mathbb Z}}
\def\X{X}
\def\Y{Y}
\def\qn{{Q^n}}  
\def\qn1{{Q^{n+1}}}
\def\d{\partial}
\def\e{\epsilon}
\def\aa{\alpha}
\def\tt{\theta}
\def\VMO{\overline{VMO}}
\def\F{\mathcal{F}}

\newtheorem{theorem}{Theorem}

\theoremstyle{remark}
\newtheorem{remark}{Remark}
\newtheorem{definition}[remark]{Definition}

\newcommand{\baa}{\begin{eqnarray}}
\newcommand{\eaa}{\end{eqnarray}}
\newcommand{\ba}{\begin{array}}
\newcommand{\ea}{\end{array}}
\newcommand{\bas}{\begin{eqnarray*}}
\newcommand{\eas}{\end{eqnarray*}}

 

\begin{center}
{\bf Well-posedness for the Navier-Stokes equations\\
Herbert Koch \\ 
Institut f\"ur Angewandte Mathematik, Universit\"at Heidelberg
}
\end{center}



%\begin{abstract}

%\end{abstract}


We study the incompressible Navier-Stokes equations on $\R^n \times
\R^+$
\begin{equation} \label{nav} 
u_t + (u \cdot \nabla ) u-\Delta u + \nabla p  =0 
\qquad 
\nabla \cdot u = 0
\end{equation}
where $u$ is the velocity and $p$ is the pressure, with inital data 
$u(x,0)= u_0(x)$.


Existence of weak solutions has been shown by Leray. Uniqueness (and regularity) of weak solutions is unknown and both are among 
the major open questions in applied analysis.
Under stronger assumptions there exist local and/or global smooth solutions.
One version of this has been shown by Kato for initial data in $L^n(\R^n)$.



What are reasonable requirements for solutions and initial data?


\begin{enumerate}

\item The nonlinearity is quadratic. To have the notion of a  weak solution 
we have to require for all $x$ and $R$
\begin{equation} \label{1}   u \in L^2(B_R(x)\times [0,R^2]) \end{equation}


\item The Navier-Stokes equations are invariant with respect to
scaling. If  $u$ is a solution (we omit the pressure in the notation,
as well as a discussion what we mean by a solution), then 
\[ \tilde u= \lambda u(\lambda x, \lambda^2 t) \]
is  again a solution. We require the   scale invariant version of 
the $L^2$ condition above
\begin{equation}\label{2}  
\sup_{x,t\le T} t^{-n/2} \int_{B_{\sqrt{t}}(x)} \int_0^{t} |u|^2 \, dy \, ds < \infty. 
\end{equation} 

\item  
The linear term should be  dominant. Let 
\begin{equation}  \Vert v \Vert_{BMO^{-1}_{T}} := \sup_{x,t\le T } \left( t^{-n/2}\int_{B_{\sqrt{t}}(x)}
\int_0^{t} |u|^2 \, dx \, ds \right)^{1/2} \end{equation} 
where $u$ is the unique function  which satisfies 
\[ u_t - \Delta u = 0, \qquad u(x,0)= v(x). \]
We define the function space $BMO^{-1}_T $ as the set of all tempered distributions for which the norm is finite.

\end{enumerate}



\begin{theorem} 
There exists $\delta>0$ with the following property: Let $0<T\le \infty$. 
If $\nabla \cdot u_0=0$ and 
$ \Vert u_0 \Vert_{(BMO^{-1}_{T})^n}
 \le \delta $ 
then there exists a unique smooth solution up to time $T$.
\end{theorem}

The theorem  implies for example existence of a unique local solution for initial velocity in $L^n$ or $L^\infty$.

The talk is based on joint work with D. Tataru. 





%\end{document}


\begin{thebibliography}{10}
  
\bibitem{benartzi} M.~Ben-Artzi,  \newblock Global solutions of
  two-dimensional {N}avier-{S}tokes and {E}uler equations.  \newblock
  {\em Arch. Rational Mech. Anal.}, 128:329--358, 1994.
  
\bibitem{brezis} H.~Brezis,  \newblock Remarks on the preceding paper
  ``{G}lobal solutions of two-dimensional {N}avier-{S}tokes and
  {E}uler equations''.  \newblock {\em Arch. Rational Mech. Anal.},
  128:359--360, 1994.
  
  
\bibitem{897.35061} M.~Cannone,  \newblock {A generalization of a
    theorem by Kato on Navier-Stokes equations.}  \newblock {\em Rev.
    Mat. Iberoam.}, 13(3):515--541, 1997.


  
\bibitem{GM} Y.~Giga and T.~Miyakawa, \newblock{Navier-Stokes flow in
    $\R\sp 3$ with measures as initial vorticity and Morrey spaces.}
  \newblock{\em Comm. Partial Differential Equations}, 14(5):
  577--618, 1989.
  
\bibitem{GMO} Y.~Giga, T.~Miyakawa and H.~Osada,
  \newblock{Two-dimensional Navier-Stokes flow with measures as
    initial vorticity.}  \newblock{\em Arch. Ration. Mech. Anal.},
  104(3): 223-250, 1988.

 
\bibitem{545.35073} T.~Kato,  \newblock {Strong $L\sp p$-solutions of
    the Navier-Stokes equation in $\R\sp m$, with applications to weak
    solutions}.  \newblock {\em {Math. Z.}}, 187:471--480, 1984.

  
\bibitem{KP} T.~Kato, and G.~ Ponce, \newblock{Commutator estimates
    and the Euler and Navier-Stokes equations.} \newblock{\em Comm.
    Pure Appl. Math.}, 41(7):891--907, 1988.


  

\bibitem{LM} P.-L.~Lions and  N.~Masmoudi, \newblock{Unicit\'e des solutions 
faibles de Navier-Stokes dans $L^N(\Omega)$} 
\newblock{\em C. R. Acad. Sci. Paris Sr. I Math}. 327 (1998), no. 5, 491--496
  
\bibitem{865.35101} F.~Planchon,  \newblock Global strong solutions in
  {S}obolev or {L}ebesgue spaces to the incompressible {Navier-Stokes}
  equations in $\mathbf{R}\sp 3$.  \newblock {\em Ann. Inst. Henri
    Poincare, Anal. Non Lineaire}, 13(3):319--336, 1996.
  

  
\bibitem{taylornav} M.~Taylor,  \newblock {Analysis on {M}orrey spaces
    and applications to {N}avier-{S}tokes equation.}  \newblock {\em
    Comm. Partial Differential Equations}, 17:1407--1456, 1992.
  

\end{thebibliography}




\end{document}


%%%%%%%%%%%%%%%%%%%%%%%%%%%%%%%%%%%%%%%%%%%%%%%%%%%%%%%%

\documentclass[12pt]{article}         
\usepackage{amsmath, amsfonts}       
\pagestyle{empty}
\title{Unique continuation for elliptic equations with nonsmooth coefficients} 
\author{Herbert Koch}   
\begin{document}                                 
\maketitle                                      


Let $P$ be a 
differential operator  and  let  $V$ and  $W$ be measurable functions. 
We consider   sufficiently regular functions $u$  which satisfy
\[  |Pu|    \le     V |u| + W |\nabla u | \]
in a ball. We say the triple $(P,V,W)$ has the  strong unique continuation property (SUCP)
if u vanishes if it vanishes of infinite order at one point.

\bigskip

{\bf
Theorem} (Koch, Tataru): Let $n > 2$ and let $P$ be an elliptic partial differential operator 
of second order with Lipschitz continuous coefficients.   
The triple $(P,V,W)$ has the SUCP 
if $V  \in L^{n/2}$  and $W \in L^{n+\epsilon}$.

\bigskip

The talk explains this result. It will  give a sketch of the proof
and explains its relation to questions occuring in the context of
nonlinear dispersive equations. 



\noindent The talk is based  on  joint work with D. Tataru.

\end{document}









